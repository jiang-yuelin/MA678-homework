% Options for packages loaded elsewhere
\PassOptionsToPackage{unicode}{hyperref}
\PassOptionsToPackage{hyphens}{url}
%
\documentclass[
]{article}
\usepackage{lmodern}
\usepackage{amssymb,amsmath}
\usepackage{ifxetex,ifluatex}
\ifnum 0\ifxetex 1\fi\ifluatex 1\fi=0 % if pdftex
  \usepackage[T1]{fontenc}
  \usepackage[utf8]{inputenc}
  \usepackage{textcomp} % provide euro and other symbols
\else % if luatex or xetex
  \usepackage{unicode-math}
  \defaultfontfeatures{Scale=MatchLowercase}
  \defaultfontfeatures[\rmfamily]{Ligatures=TeX,Scale=1}
\fi
% Use upquote if available, for straight quotes in verbatim environments
\IfFileExists{upquote.sty}{\usepackage{upquote}}{}
\IfFileExists{microtype.sty}{% use microtype if available
  \usepackage[]{microtype}
  \UseMicrotypeSet[protrusion]{basicmath} % disable protrusion for tt fonts
}{}
\makeatletter
\@ifundefined{KOMAClassName}{% if non-KOMA class
  \IfFileExists{parskip.sty}{%
    \usepackage{parskip}
  }{% else
    \setlength{\parindent}{0pt}
    \setlength{\parskip}{6pt plus 2pt minus 1pt}}
}{% if KOMA class
  \KOMAoptions{parskip=half}}
\makeatother
\usepackage{xcolor}
\IfFileExists{xurl.sty}{\usepackage{xurl}}{} % add URL line breaks if available
\IfFileExists{bookmark.sty}{\usepackage{bookmark}}{\usepackage{hyperref}}
\hypersetup{
  pdftitle={MA678 Homework 4},
  pdfauthor={Name},
  hidelinks,
  pdfcreator={LaTeX via pandoc}}
\urlstyle{same} % disable monospaced font for URLs
\usepackage[margin=1in]{geometry}
\usepackage{color}
\usepackage{fancyvrb}
\newcommand{\VerbBar}{|}
\newcommand{\VERB}{\Verb[commandchars=\\\{\}]}
\DefineVerbatimEnvironment{Highlighting}{Verbatim}{commandchars=\\\{\}}
% Add ',fontsize=\small' for more characters per line
\usepackage{framed}
\definecolor{shadecolor}{RGB}{248,248,248}
\newenvironment{Shaded}{\begin{snugshade}}{\end{snugshade}}
\newcommand{\AlertTok}[1]{\textcolor[rgb]{0.94,0.16,0.16}{#1}}
\newcommand{\AnnotationTok}[1]{\textcolor[rgb]{0.56,0.35,0.01}{\textbf{\textit{#1}}}}
\newcommand{\AttributeTok}[1]{\textcolor[rgb]{0.77,0.63,0.00}{#1}}
\newcommand{\BaseNTok}[1]{\textcolor[rgb]{0.00,0.00,0.81}{#1}}
\newcommand{\BuiltInTok}[1]{#1}
\newcommand{\CharTok}[1]{\textcolor[rgb]{0.31,0.60,0.02}{#1}}
\newcommand{\CommentTok}[1]{\textcolor[rgb]{0.56,0.35,0.01}{\textit{#1}}}
\newcommand{\CommentVarTok}[1]{\textcolor[rgb]{0.56,0.35,0.01}{\textbf{\textit{#1}}}}
\newcommand{\ConstantTok}[1]{\textcolor[rgb]{0.00,0.00,0.00}{#1}}
\newcommand{\ControlFlowTok}[1]{\textcolor[rgb]{0.13,0.29,0.53}{\textbf{#1}}}
\newcommand{\DataTypeTok}[1]{\textcolor[rgb]{0.13,0.29,0.53}{#1}}
\newcommand{\DecValTok}[1]{\textcolor[rgb]{0.00,0.00,0.81}{#1}}
\newcommand{\DocumentationTok}[1]{\textcolor[rgb]{0.56,0.35,0.01}{\textbf{\textit{#1}}}}
\newcommand{\ErrorTok}[1]{\textcolor[rgb]{0.64,0.00,0.00}{\textbf{#1}}}
\newcommand{\ExtensionTok}[1]{#1}
\newcommand{\FloatTok}[1]{\textcolor[rgb]{0.00,0.00,0.81}{#1}}
\newcommand{\FunctionTok}[1]{\textcolor[rgb]{0.00,0.00,0.00}{#1}}
\newcommand{\ImportTok}[1]{#1}
\newcommand{\InformationTok}[1]{\textcolor[rgb]{0.56,0.35,0.01}{\textbf{\textit{#1}}}}
\newcommand{\KeywordTok}[1]{\textcolor[rgb]{0.13,0.29,0.53}{\textbf{#1}}}
\newcommand{\NormalTok}[1]{#1}
\newcommand{\OperatorTok}[1]{\textcolor[rgb]{0.81,0.36,0.00}{\textbf{#1}}}
\newcommand{\OtherTok}[1]{\textcolor[rgb]{0.56,0.35,0.01}{#1}}
\newcommand{\PreprocessorTok}[1]{\textcolor[rgb]{0.56,0.35,0.01}{\textit{#1}}}
\newcommand{\RegionMarkerTok}[1]{#1}
\newcommand{\SpecialCharTok}[1]{\textcolor[rgb]{0.00,0.00,0.00}{#1}}
\newcommand{\SpecialStringTok}[1]{\textcolor[rgb]{0.31,0.60,0.02}{#1}}
\newcommand{\StringTok}[1]{\textcolor[rgb]{0.31,0.60,0.02}{#1}}
\newcommand{\VariableTok}[1]{\textcolor[rgb]{0.00,0.00,0.00}{#1}}
\newcommand{\VerbatimStringTok}[1]{\textcolor[rgb]{0.31,0.60,0.02}{#1}}
\newcommand{\WarningTok}[1]{\textcolor[rgb]{0.56,0.35,0.01}{\textbf{\textit{#1}}}}
\usepackage{graphicx,grffile}
\makeatletter
\def\maxwidth{\ifdim\Gin@nat@width>\linewidth\linewidth\else\Gin@nat@width\fi}
\def\maxheight{\ifdim\Gin@nat@height>\textheight\textheight\else\Gin@nat@height\fi}
\makeatother
% Scale images if necessary, so that they will not overflow the page
% margins by default, and it is still possible to overwrite the defaults
% using explicit options in \includegraphics[width, height, ...]{}
\setkeys{Gin}{width=\maxwidth,height=\maxheight,keepaspectratio}
% Set default figure placement to htbp
\makeatletter
\def\fps@figure{htbp}
\makeatother
\setlength{\emergencystretch}{3em} % prevent overfull lines
\providecommand{\tightlist}{%
  \setlength{\itemsep}{0pt}\setlength{\parskip}{0pt}}
\setcounter{secnumdepth}{-\maxdimen} % remove section numbering

\title{MA678 Homework 4}
\author{Name}
\date{}

\begin{document}
\maketitle

\hypertarget{disclaimer}{%
\subsection{Disclaimer}\label{disclaimer}}

A few things to keep in mind :\\
1) Use set.seed() to make sure that the document produces the same
random simulation as when you ran the code.\\
2) Use refresh=0 for any stan\_glm() or stan-based model. lm() or
non-stan models don't need this!\\
3) You can type outside of the r chunks and make new r chunks where it's
convenient. Make sure it's clear which questions you're answering.\\
4) Even if you're not too confident, please try giving an answer to the
text responses!\\
5) Please don't print data in the document unless the question asks.
It's good for you to do it to look at the data, but not as good for
someone trying to read the document later on.\\
6) Check your document before submitting! Please put your name where
``name'' is by the author!

\hypertarget{section}{%
\section{13.5}\label{section}}

Interpreting logistic regression coefficients: Here is a fitted model
from the Bangladesh analysis predicting whether a person with
high-arsenic drinking water will switch wells, given the arsenic level
in their existing well and the distance to the nearest safe well:

stan\_glm(formula = switch \textasciitilde{} dist100 + arsenic,
family=binomial(link=``logit''), data=wells)

Median MAD\_SD\\
(Intercept) 0.00 0.08\\
dist100 -0.90 0.10\\
arsenic 0.46 0.04

Compare two people who live the same distance from the nearest well but
whose arsenic levels differ, with one person having an arsenic level of
0.5 and the other person having a level of 1.0. You will estimate how
much more likely this second person is to switch wells. Give an
approximate estimate, standard error, 50\% interval, and 95\% interval,
using two different methods:

\hypertarget{a}{%
\subsection{(a)}\label{a}}

Use the divide-by-4 rule, based on the information from this regression
output.\\
Answer: Pr(0.5 arsenic) = invlogit(0 + 0.46 * 0.5). Divide 0.46 by 4 we
get 0.115, so 1 difference in arsenic level corresponds to a 11.5\%
probability of switching. With standard error 0.04, so 95\% range of the
coefficient of arsenic would be {[}0.115 +- 0.04 * 2{]} = {[}0.035,
0.195{]}. And a difference of 0.5 would be estimated 5.75\% more likely
of switching, 95\% range would be {[}0.018,0.098{]}.

\hypertarget{b}{%
\subsection{(b)}\label{b}}

Use predictive simulation from the fitted model in R, under the
assumption that these two people each live 50 meters from the nearest
safe well.

\begin{Shaded}
\begin{Highlighting}[]
\KeywordTok{set.seed}\NormalTok{(}\DecValTok{100}\NormalTok{)}
\NormalTok{wells <-}\StringTok{ }\KeywordTok{read.csv}\NormalTok{(}\StringTok{"wells.csv"}\NormalTok{)}
\NormalTok{fit1 <-}\StringTok{ }\KeywordTok{stan_glm}\NormalTok{(}\DataTypeTok{formula =} \ControlFlowTok{switch} \OperatorTok{~}\StringTok{ }\NormalTok{dist100 }\OperatorTok{+}\StringTok{ }\NormalTok{arsenic, }\DataTypeTok{family=}\KeywordTok{binomial}\NormalTok{(}\DataTypeTok{link=}\StringTok{"logit"}\NormalTok{),  }\DataTypeTok{data=}\NormalTok{wells, }\DataTypeTok{refresh =} \DecValTok{0}\NormalTok{)}
\KeywordTok{print}\NormalTok{(fit1, }\DataTypeTok{digits =}\DecValTok{2}\NormalTok{)}
\end{Highlighting}
\end{Shaded}

\begin{verbatim}
## stan_glm
##  family:       binomial [logit]
##  formula:      switch ~ dist100 + arsenic
##  observations: 3020
##  predictors:   3
## ------
##             Median MAD_SD
## (Intercept)  0.00   0.08 
## dist100     -0.90   0.10 
## arsenic      0.46   0.04 
## 
## ------
## * For help interpreting the printed output see ?print.stanreg
## * For info on the priors used see ?prior_summary.stanreg
\end{verbatim}

\begin{Shaded}
\begin{Highlighting}[]
\NormalTok{new <-}\StringTok{ }\KeywordTok{data.frame}\NormalTok{(}\DataTypeTok{dist100 =} \FloatTok{0.5}\NormalTok{, }\DataTypeTok{arsenic =} \FloatTok{0.5}\NormalTok{)}
\NormalTok{pred <-}\StringTok{ }\KeywordTok{predict}\NormalTok{(fit1, }\DataTypeTok{newdata =}\NormalTok{ new )}
\NormalTok{new1 <-}\StringTok{ }\KeywordTok{data.frame}\NormalTok{(}\DataTypeTok{dist100 =} \FloatTok{0.5}\NormalTok{, }\DataTypeTok{arsenic =} \DecValTok{1}\NormalTok{)}
\NormalTok{pred1 <-}\StringTok{ }\KeywordTok{predict}\NormalTok{(fit1, }\DataTypeTok{newdata =}\NormalTok{ new1 )}
\KeywordTok{print}\NormalTok{(}\KeywordTok{paste}\NormalTok{(}\StringTok{"pred1 - pred="}\NormalTok{,pred1}\OperatorTok{-}\NormalTok{pred))}
\end{Highlighting}
\end{Shaded}

\begin{verbatim}
## [1] "pred1 - pred= 0.231050322381434"
\end{verbatim}

Answer: So the second person are 23\% more likely to switch well than
the first person.

\hypertarget{section-1}{%
\section{13.7}\label{section-1}}

Graphing a fitted logistic regression: We downloaded data with weight
(in pounds) and age (in years) from a random sample of American adults.
We then defined a new variable: heavy \textless- weight \textgreater{}
200 and fit a logistic regression, predicting heavy from height (in
inches):\\
stan\_glm(formula = heavy \textasciitilde{} height,
family=binomial(link=``logit''), data=health)\\
Median MAD\_SD\\
(Intercept) -21.51 1.60\\
height 0.28 0.02

\hypertarget{a-1}{%
\subsection{(a)}\label{a-1}}

Graph the logistic regression curve (the probability that someone is
heavy) over the approximate range of the data. Be clear where the line
goes through the 50\% probability point.

\hypertarget{b-1}{%
\subsection{(b)}\label{b-1}}

Fill in the blank: near the 50\% point, comparing two people who differ
by one inch in height, you'll expect a difference of \_\_\_\_ in the
probability of being heavy. Answer: 7\%

\hypertarget{section-2}{%
\section{13.8}\label{section-2}}

Linear transformations: In the regression from the previous exercise,
suppose you replaced height in inches by height in centimeters. What
would then be the intercept and slope? Answer: The intercept does not
change and the slope change to 2.5 times prior.

\hypertarget{section-3}{%
\section{13.10}\label{section-3}}

Expressing a comparison of proportions as a logistic regression: A
randomized experiment is performed within a survey, and 1000 people are
contacted. Half the people contacted are promised a \$5 incentive to
participate, and half are not promised an incentive. The result is a
50\% response rate among the treated group and 40\% response rate among
the control group.

\hypertarget{a-2}{%
\subsection{(a)}\label{a-2}}

Set up these results as data in R. From these data, fit a logistic
regression of response on the treatment indicator.

\begin{Shaded}
\begin{Highlighting}[]
\KeywordTok{set.seed}\NormalTok{(}\DecValTok{1000}\NormalTok{)}
\NormalTok{tgroup <-}\StringTok{ }\KeywordTok{data.frame}\NormalTok{(}\DataTypeTok{x =}\DecValTok{1}\NormalTok{, }\DataTypeTok{y=}\KeywordTok{rbinom}\NormalTok{(}\DecValTok{500}\NormalTok{, }\DecValTok{1}\NormalTok{, }\FloatTok{0.5}\NormalTok{))}
\NormalTok{cgroup <-}\StringTok{ }\KeywordTok{data.frame}\NormalTok{(}\DataTypeTok{x=} \DecValTok{0}\NormalTok{, }\DataTypeTok{y=}\KeywordTok{rbinom}\NormalTok{(}\DecValTok{500}\NormalTok{, }\DecValTok{1}\NormalTok{, }\FloatTok{0.4}\NormalTok{))}
\NormalTok{data1 <-}\StringTok{ }\KeywordTok{rbind}\NormalTok{(tgroup, cgroup)}
\NormalTok{fit2 <-}\StringTok{ }\KeywordTok{stan_glm}\NormalTok{(y}\OperatorTok{~}\NormalTok{x, }\DataTypeTok{family =} \KeywordTok{binomial}\NormalTok{(}\DataTypeTok{link =} \StringTok{"logit"}\NormalTok{), }\DataTypeTok{data =}\NormalTok{ data1, }\DataTypeTok{refresh =}\DecValTok{0}\NormalTok{)}
\KeywordTok{print}\NormalTok{(fit2, }\DataTypeTok{digits =} \DecValTok{2}\NormalTok{)}
\end{Highlighting}
\end{Shaded}

\begin{verbatim}
## stan_glm
##  family:       binomial [logit]
##  formula:      y ~ x
##  observations: 1000
##  predictors:   2
## ------
##             Median MAD_SD
## (Intercept) -0.48   0.09 
## x            0.65   0.13 
## 
## ------
## * For help interpreting the printed output see ?print.stanreg
## * For info on the priors used see ?prior_summary.stanreg
\end{verbatim}

\hypertarget{b-2}{%
\subsection{(b)}\label{b-2}}

Compare to the results from Exercise 4.1.

\hypertarget{section-4}{%
\section{13.11}\label{section-4}}

Building a logistic regression model: The folder Rodents contains data
on rodents in a sample of New York City apartments.

\hypertarget{a-3}{%
\subsection{(a)}\label{a-3}}

Build a logistic regression model to predict the presence of rodents
(the variable rodent2 in the dataset) given indicators for the ethnic
groups (race). Combine categories as appropriate. Discuss the estimated
coefficients in the model.

\begin{Shaded}
\begin{Highlighting}[]
\KeywordTok{set.seed}\NormalTok{(}\DecValTok{100}\NormalTok{)}
\NormalTok{rod <-}\StringTok{ }\KeywordTok{read.csv}\NormalTok{(}\StringTok{"hvs02_sorted.csv"}\NormalTok{)}
\CommentTok{# newrod <- fct_collapse(rod$race, white=0, poc =c(2,3,4,5,6,7))}

\CommentTok{# combine groups of race: white = 0, poc/multi racial = 1}
\ControlFlowTok{for}\NormalTok{ (i }\ControlFlowTok{in} \DecValTok{1}\OperatorTok{:}\KeywordTok{length}\NormalTok{(rod}\OperatorTok{$}\NormalTok{race)) \{}
  \ControlFlowTok{if}\NormalTok{(rod}\OperatorTok{$}\NormalTok{race[i] }\OperatorTok{==}\DecValTok{1}\NormalTok{)\{}
\NormalTok{    rod}\OperatorTok{$}\NormalTok{race[i] <-}\StringTok{ }\DecValTok{0}
\NormalTok{  \} }\ControlFlowTok{else}\NormalTok{\{}
\NormalTok{    rod}\OperatorTok{$}\NormalTok{race[i] <-}\StringTok{ }\DecValTok{1}
\NormalTok{  \}}
\NormalTok{\}}
\CommentTok{# fit the logistic regression on a single variable race.}
\NormalTok{fit3 <-}\StringTok{ }\KeywordTok{stan_glm}\NormalTok{(rodent2 }\OperatorTok{~}\StringTok{ }\NormalTok{race , }\DataTypeTok{family =} \KeywordTok{binomial}\NormalTok{(}\DataTypeTok{link =} \StringTok{"logit"}\NormalTok{), }\DataTypeTok{data =}\NormalTok{ rod, }\DataTypeTok{refresh =}\DecValTok{0}\NormalTok{)}
\KeywordTok{print}\NormalTok{(fit3)}
\end{Highlighting}
\end{Shaded}

\begin{verbatim}
## stan_glm
##  family:       binomial [logit]
##  formula:      rodent2 ~ race
##  observations: 13931
##  predictors:   2
## ------
##             Median MAD_SD
## (Intercept) -2.2    0.0  
## race         1.5    0.0  
## 
## ------
## * For help interpreting the printed output see ?print.stanreg
## * For info on the priors used see ?prior_summary.stanreg
\end{verbatim}

\hypertarget{b-3}{%
\subsection{(b)}\label{b-3}}

Add to your model some other potentially relevant predictors describing
the apartment, building, and community district. Build your model using
the general principles explained in Section 12.6. Discuss the
coefficients for the ethnicity indicators in your model.

\begin{Shaded}
\begin{Highlighting}[]
\CommentTok{# add new variables to the logistic model: }
 \CommentTok{# race: binary variable, people of color =1, white = 0}
 \CommentTok{# }
\KeywordTok{set.seed}\NormalTok{(}\DecValTok{100}\NormalTok{)}
\NormalTok{fit4 <-}\StringTok{ }\KeywordTok{stan_glm}\NormalTok{(rodent2 }\OperatorTok{~}\StringTok{ }\NormalTok{race }\OperatorTok{+}\StringTok{ }\NormalTok{intcrack2 }\OperatorTok{+}\NormalTok{inthole2 }\OperatorTok{+}\StringTok{ }\NormalTok{race}\OperatorTok{:}\NormalTok{povertyx2, }\DataTypeTok{family =} \KeywordTok{binomial}\NormalTok{(}\DataTypeTok{link =} \StringTok{"logit"}\NormalTok{), }\DataTypeTok{data =}\NormalTok{ rod, }\DataTypeTok{refresh =}\DecValTok{0}\NormalTok{)}
\KeywordTok{print}\NormalTok{(fit4, }\DataTypeTok{digits =}\DecValTok{2}\NormalTok{)}
\end{Highlighting}
\end{Shaded}

\begin{verbatim}
## stan_glm
##  family:       binomial [logit]
##  formula:      rodent2 ~ race + intcrack2 + inthole2 + race:povertyx2
##  observations: 13510
##  predictors:   5
## ------
##                Median MAD_SD
## (Intercept)    -2.42   0.05 
## race            1.13   0.06 
## intcrack2       1.31   0.06 
## inthole2        1.30   0.09 
## race:povertyx2  0.44   0.05 
## 
## ------
## * For help interpreting the printed output see ?print.stanreg
## * For info on the priors used see ?prior_summary.stanreg
\end{verbatim}

\hypertarget{section-5}{%
\section{14.3}\label{section-5}}

Graphing logistic regressions: The well-switching data described in
Section 13.7 are in the folder Arsenic.

\hypertarget{a-4}{%
\subsection{(a)}\label{a-4}}

Fit a logistic regression for the probability of switching using log
(distance to nearest safe well) as a predictor.

\begin{Shaded}
\begin{Highlighting}[]
\KeywordTok{set.seed}\NormalTok{(}\DecValTok{100}\NormalTok{)}
\NormalTok{fit5 <-}\StringTok{ }\KeywordTok{stan_glm}\NormalTok{(}\ControlFlowTok{switch} \OperatorTok{~}\StringTok{ }\NormalTok{dist,}\DataTypeTok{family =} \KeywordTok{binomial}\NormalTok{(}\DataTypeTok{link =} \StringTok{"logit"}\NormalTok{) ,}\DataTypeTok{data =}\NormalTok{ wells, }\DataTypeTok{refresh =} \DecValTok{0}\NormalTok{)}
\KeywordTok{print}\NormalTok{(fit5)}
\end{Highlighting}
\end{Shaded}

\begin{verbatim}
## stan_glm
##  family:       binomial [logit]
##  formula:      switch ~ dist
##  observations: 3020
##  predictors:   2
## ------
##             Median MAD_SD
## (Intercept) 0.6    0.1   
## dist        0.0    0.0   
## 
## ------
## * For help interpreting the printed output see ?print.stanreg
## * For info on the priors used see ?prior_summary.stanreg
\end{verbatim}

\hypertarget{b-4}{%
\subsection{(b)}\label{b-4}}

Make a graph similar to Figure 13.8b displaying Pr(switch) as a function
of distance to nearest safe well, along with the data.

\begin{Shaded}
\begin{Highlighting}[]
\CommentTok{# Jitter binary outcome variable switch between 0 and 1}
\KeywordTok{set.seed}\NormalTok{(}\DecValTok{100}\NormalTok{)}
\NormalTok{jitter_bi <-}\StringTok{ }\ControlFlowTok{function}\NormalTok{(a, }\DataTypeTok{jitt =} \FloatTok{0.05}\NormalTok{)\{}
  \KeywordTok{ifelse}\NormalTok{(a}\OperatorTok{==}\DecValTok{0}\NormalTok{, }\KeywordTok{runif}\NormalTok{(}\KeywordTok{length}\NormalTok{(a), }\DecValTok{0}\NormalTok{, jitt), }\KeywordTok{runif}\NormalTok{(}\KeywordTok{length}\NormalTok{(a), }\DecValTok{1}\OperatorTok{-}\NormalTok{jitt, }\DecValTok{1}\NormalTok{))}
\NormalTok{\}}

\NormalTok{wells}\OperatorTok{$}\NormalTok{switch_jitter <-}\StringTok{ }\KeywordTok{jitter_bi}\NormalTok{(wells}\OperatorTok{$}\ControlFlowTok{switch}\NormalTok{)}
\KeywordTok{plot}\NormalTok{(wells}\OperatorTok{$}\NormalTok{dist, wells}\OperatorTok{$}\NormalTok{switch_jitter)}
\KeywordTok{curve}\NormalTok{(}\KeywordTok{invlogit}\NormalTok{(}\KeywordTok{coef}\NormalTok{(fit5)[}\DecValTok{1}\NormalTok{] }\OperatorTok{+}\StringTok{ }\KeywordTok{coef}\NormalTok{(fit5)[}\DecValTok{2}\NormalTok{]}\OperatorTok{*}\NormalTok{x), }\DataTypeTok{add =} \OtherTok{TRUE}\NormalTok{)}
\end{Highlighting}
\end{Shaded}

\includegraphics{HW4-Blank-1-_files/figure-latex/unnamed-chunk-6-1.pdf}

\hypertarget{c}{%
\subsection{(c)}\label{c}}

Make a residual plot and binned residual plot as in Figure 14.8.

\begin{Shaded}
\begin{Highlighting}[]
\KeywordTok{set.seed}\NormalTok{(}\DecValTok{100}\NormalTok{)}
\KeywordTok{par}\NormalTok{(}\DataTypeTok{mfrow=}\KeywordTok{c}\NormalTok{(}\DecValTok{2}\NormalTok{,}\DecValTok{2}\NormalTok{))}

\KeywordTok{plot}\NormalTok{(}\KeywordTok{predict}\NormalTok{(fit5), }\KeywordTok{residuals}\NormalTok{(fit5), }\DataTypeTok{main=}\StringTok{"Residual plot"}\NormalTok{, }\DataTypeTok{xlab=}\StringTok{"Expected Values"}\NormalTok{, }\DataTypeTok{ylab=}\StringTok{"Residuals"}\NormalTok{)}
\KeywordTok{binnedplot}\NormalTok{(}\KeywordTok{predict}\NormalTok{(fit5) , }\KeywordTok{residuals}\NormalTok{(fit5))}
\end{Highlighting}
\end{Shaded}

\includegraphics{HW4-Blank-1-_files/figure-latex/unnamed-chunk-7-1.pdf}

\hypertarget{d}{%
\subsection{(d)}\label{d}}

Compute the error rate of the fitted model and compare to the error rate
of the null model.

\begin{Shaded}
\begin{Highlighting}[]
\KeywordTok{set.seed}\NormalTok{(}\DecValTok{100}\NormalTok{)}
\CommentTok{# error rate of the fitted model}
\NormalTok{predicted <-}\StringTok{ }\KeywordTok{predict}\NormalTok{(fit5)}
\NormalTok{y <-}\StringTok{ }\NormalTok{wells}\OperatorTok{$}\ControlFlowTok{switch}
\KeywordTok{mean}\NormalTok{((predicted}\OperatorTok{>}\FloatTok{0.5} \OperatorTok{&}\StringTok{ }\NormalTok{y}\OperatorTok{==}\DecValTok{0}\NormalTok{) }\OperatorTok{|}\StringTok{ }\NormalTok{(predicted}\OperatorTok{<}\FloatTok{0.5} \OperatorTok{&}\StringTok{ }\NormalTok{y}\OperatorTok{==}\DecValTok{1}\NormalTok{))}
\end{Highlighting}
\end{Shaded}

\begin{verbatim}
## [1] 0.5387417
\end{verbatim}

\begin{Shaded}
\begin{Highlighting}[]
\CommentTok{# error rate of the null model}
\NormalTok{predicted.null <-}\StringTok{ }\KeywordTok{seq}\NormalTok{(}\DecValTok{0}\NormalTok{, }\DecValTok{0}\NormalTok{, }\DataTypeTok{length.out=}\KeywordTok{length}\NormalTok{(y))}
\KeywordTok{mean}\NormalTok{((predicted.null}\OperatorTok{>}\FloatTok{0.5} \OperatorTok{&}\StringTok{ }\NormalTok{y}\OperatorTok{==}\DecValTok{0}\NormalTok{) }\OperatorTok{|}\StringTok{ }\NormalTok{(predicted.null}\OperatorTok{<}\FloatTok{0.5} \OperatorTok{&}\StringTok{ }\NormalTok{y}\OperatorTok{==}\DecValTok{1}\NormalTok{))}
\end{Highlighting}
\end{Shaded}

\begin{verbatim}
## [1] 0.5751656
\end{verbatim}

\hypertarget{e}{%
\subsection{(e)}\label{e}}

Create indicator variables corresponding to dist\textless100; dist
between 100 and 200; and dist\textgreater200. Fit a logistic regression
for Pr(switch) using these indicators. With this new model, repeat the
computations and graphs for part (a) of this exercise.

\begin{Shaded}
\begin{Highlighting}[]
\NormalTok{wells}\OperatorTok{$}\NormalTok{dist_lt100 <-}\StringTok{ }\KeywordTok{as.numeric}\NormalTok{(wells}\OperatorTok{$}\NormalTok{dist }\OperatorTok{<}\StringTok{ }\DecValTok{100}\NormalTok{)}
\NormalTok{wells}\OperatorTok{$}\NormalTok{dist_gte100_lt200 <-}\StringTok{ }\KeywordTok{as.numeric}\NormalTok{(}\DecValTok{100} \OperatorTok{<=}\StringTok{ }\NormalTok{wells}\OperatorTok{$}\NormalTok{dist }\OperatorTok{&}\StringTok{ }\NormalTok{wells}\OperatorTok{$}\NormalTok{dist }\OperatorTok{<}\StringTok{ }\DecValTok{200}\NormalTok{)}
\NormalTok{wells}\OperatorTok{$}\NormalTok{dist_gte200 <-}\StringTok{ }\KeywordTok{as.numeric}\NormalTok{(wells}\OperatorTok{$}\NormalTok{dist }\OperatorTok{<=}\StringTok{ }\DecValTok{200}\NormalTok{)}
\NormalTok{fit6 <-}\StringTok{ }\KeywordTok{glm}\NormalTok{(}\ControlFlowTok{switch} \OperatorTok{~}\StringTok{ }\NormalTok{dist_lt100 }\OperatorTok{+}\StringTok{ }\NormalTok{dist_gte100_lt200 }\OperatorTok{+}\StringTok{ }\NormalTok{dist_gte200, }\DataTypeTok{data=}\NormalTok{wells, }\DataTypeTok{family=}\KeywordTok{binomial}\NormalTok{(}\DataTypeTok{link=}\StringTok{"logit"}\NormalTok{))}
\KeywordTok{display}\NormalTok{(fit6)}
\end{Highlighting}
\end{Shaded}

\begin{verbatim}
## glm(formula = switch ~ dist_lt100 + dist_gte100_lt200 + dist_gte200, 
##     family = binomial(link = "logit"), data = wells)
##                   coef.est coef.se
## (Intercept)       -1.25     0.80  
## dist_lt100         1.63     0.80  
## dist_gte100_lt200  0.97     0.81  
## ---
##   n = 3020, k = 3
##   residual deviance = 4084.7, null deviance = 4118.1 (difference = 33.4)
\end{verbatim}

\#14.5 Working with logistic regression: In a class of 50 students, a
logistic regression is performed of course grade (pass or fail) on
midterm exam score (continuous values with mean 60 and standard
deviation 15). The fitted model is Pr(pass) = logit-1(-24 + 0.4x).

\hypertarget{a-5}{%
\subsection{(a)}\label{a-5}}

Graph the fitted model. Also on this graph put a scatterplot of
hypothetical data consistent with the information given.

\hypertarget{b-5}{%
\subsection{(b)}\label{b-5}}

Suppose the midterm scores were transformed to have a mean of 0 and
standard deviation of 1. What would be the equation of the logistic
regression using these transformed scores as a redictor?

\hypertarget{c-1}{%
\subsection{(c)}\label{c-1}}

Create a new predictor that is pure noise; for example, in R you can
create newpred \textless- rnorm(n,0,1). Add it to your model. How much
does the leave-one-out cross validation score decrease?

\#14.7 Model building and comparison: Continue with the well-switching
data described in the previous exercise.

\hypertarget{a-6}{%
\subsection{(a)}\label{a-6}}

Fit a logistic regression for the probability of switching using, as
predictors, distance, log(arsenic), and their interaction. Interpret the
estimated coefficients and their standard errors.

\hypertarget{b-6}{%
\subsection{(b)}\label{b-6}}

Make graphs as in Figure 14.3 to show the relation between probability
of switching, distance, and arsenic level.

\hypertarget{c-2}{%
\subsection{(c)}\label{c-2}}

Following the procedure described in Section 14.4, compute the average
predictive differences corresponding to:\\
i. A comparison of dist = 0 to dist = 100, with arsenic held constant.\\
ii. A comparison of dist = 100 to dist = 200, with arsenic held
constant.\\
iii. A comparison of arsenic = 0.5 to arsenic = 1.0, with dist held
constant.\\
iv. A comparison of arsenic = 1.0 to arsenic = 2.0, with dist held
constant.

Discuss these results.

\hypertarget{section-6}{%
\section{14.9}\label{section-6}}

Linear or logistic regression for discrete data: Simulate continuous
data from the regression model, z = a + bx + error. Set the parameters
so that the outcomes z are positive about half the time and negative
about half the time.

\hypertarget{a-7}{%
\subsection{(a)}\label{a-7}}

Create a binary variable y that equals 1 if z is positive or 0 if z is
negative. Fit a logistic regression predicting y from x.

\hypertarget{b-7}{%
\subsection{(b)}\label{b-7}}

Fit a linear regression predicting y from x: you can do this, even
though the data y are discrete.

\hypertarget{c-3}{%
\subsection{(c)}\label{c-3}}

Estimate the average predictive comparison---the expected difference in
y, corresponding to a unit difference in x---based on the fitted
logistic regression in (a). Compare this average predictive comparison
to the linear regression coefficient in (b).

\hypertarget{section-7}{%
\section{14.10}\label{section-7}}

Linear or logistic regression for discrete data: In the setup of the
previous exercise:

\hypertarget{a-8}{%
\subsection{(a)}\label{a-8}}

Set the parameters of your simulation so that the coefficient estimate
in (b) and the average predictive comparison in (c) are close.

\hypertarget{b-8}{%
\subsection{(b)}\label{b-8}}

Set the parameters of your simulation so that the coefficient estimate
in (b) and the average predictive comparison in (c) are much different.

\hypertarget{c-4}{%
\subsection{(c)}\label{c-4}}

In general, when will it work reasonably well to fit a linear model to
predict a binary outcome? See also Exercise 13.12.

\end{document}
